\chapter{Outcomes}

\section{Learning Outcomes} % .5 pages, maybe less
At the start of the project we naively assumed that street lights and lightning would be key explanatory features for graffiti, and started working on the assumption that this would be our most important feature to consider. This proved not to be the case, but luckily we had considered adding additional features from available city data from the start. Pivoting to more comprehensive predictions on city layouts was feasible, and reminded us to not underestimate the scope of a project and potential required variables.

With the models, we came to an important revelation that producing a single model that would perform the best in all situations is not always feasible. Insted, a model that performs better in eliminating false positives can be more suitable in some situations, and one eliminating false negatives in others. Provided these features are communicated clearly and understood. 

We also gained useful insight and experience in creating map visualizations and presenting geographical data, and understood its importance. Many of our key findings were made by plotting features on a coordinate space and contrasting these with real maps of Vancouver.

%
