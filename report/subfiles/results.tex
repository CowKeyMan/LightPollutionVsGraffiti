\chapter{Outcomes}
In this section we will discuss our findings and how they are presented to the target audience. Our learning outcomes are introduced as well, in particular how and why our initial project scope changed during the project and what key insigths we made while working on it. 

\section{Presenting Results}

Since this is mostly a research project, our results are presented as a \href{https://cowkeyman.github.io/PredictingGraffitiUsingCityLayouts/}{blog}. The main target audience is the Vancouver city council, as we are performing this project on their dataset. The secondadry main target audience is other city councils who wish to reduce graffiti in their cities. This project may also prove valuable to city planners for utilizing our data to design graffiti-proof cities. Condsideration has been taken for researchers to use our work as inspiration and a starting point on the topic.

In the blog, we introduce the problem, show some initial observations obtained from the data and then present our findings after fitting the models. In order to identify features which prevent or encourage graffiti, we explain our linear models in a way understandable both to a professionals as well as those who may poses less technical knowledge of used methods. Models are used to showcase different predictions on where new graffiti might crop up. Differences between used methods are analysed as well, notably between regression and classification. For classification, we use two variations of Logistic Regression, with balanced and unbalanced weights. These classification models cater for different use cases and preventative plans.  

\section{Learning Outcomes} % .5 pages, maybe less
At the start of the project we naively assumed that street lights and lightning would be key explanatory features for graffiti, and started working on the assumption that this would be our most important feature to consider. This proved not to be the case, but luckily we had considered adding additional features from available city data from the start. Pivoting to more comprehensive predictions on city layouts was feasible, and reminded us to not underestimate the scope of a project and potential required variables.

With the models, we came to an important revelation that producing a single model that would perform the best in all situations is not always feasible. Insted, a model that performs better in eliminating false positives can be more suitable in some situations, and one eliminating false negatives in others. Provided these features are communicated clearly and understood. 

We also gained useful insight and experience in creating map visualizations and presenting geographical data, and understood its importance. Many of our key findings were made by plotting features on a coordinate space and contrasting these with real maps of Vancouver.


