\begin{abstract}

Graffiti is a global problem that is costing governments billions of dollars each year for cleanup and repair costs. The prevalence of graffiti can also lead to other negative effects such as discouraging tourism and causing damage to affected surfaces. In this project, we conduct what is, to the best of our knowledge, the first ever study about finding methods to predict and prevent graffiti using the features and layout of a city. Our methods are preliminary, but show promising results. While future work may uncover better features in predicting this act of vandalism, we provide a good starting point that can already be used to show hypothetical relationships between the chosen city features and the generation of graffiti.

\end{abstract}
