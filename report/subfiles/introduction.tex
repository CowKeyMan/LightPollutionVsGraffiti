\chapter{Introduction}

% Define problem
% What we started with and initial problem
% What we are ended up doing (briefly)
% Say outline of report

Graffiti is an act of vandalism, most commonly encountered in the form of defacing buildings and public surfaces with spray paint. Graffiti is often seen as unpleasant, evoking feelings of disarray and public unrest. The paint can also be harmful to building surfaces. Removing graffiti is often expensive, which is why preventative measures are preferable. However, in many cases these preventative methods are actually reactive, trying to prevent further graffiti in a vandalised area. Little research has been done on what causes graffiti and how it could be prevented.

Our research aims to understand the causes of graffiti on a more fundamental level to build a predictive model to help prevent further graffiti. Features found in city layouts are analysed for potential correlations with graffiti, and are tested in different machine learning models to find best predictive accuracy. These predictions could then be potentially used, for example, to identify areas with high risks of graffiti or help with cleaning efforts. Additionally we will try to build an explainable model to illustrate what features in city layouts correlate with graffiti, with the goal of providing information for city planners to plan for graffiti-proof cities. 

Methodology is reported in the first chapter. Used datasets are introduced briefly, as well as any applied data cleaning and processing. Build models and their features are also discussed in this chapter, together with a section on how the final results are presented to the target audience. In the following chapter our findings and results are analysed, and observations are made on our learning outcomes during the project. The report concludes by summarizing our findings and considering possible further research.