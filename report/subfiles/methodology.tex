\chapter{Methodology}

% mention how graffiti with light would not work, say why we opted for buildings

This chapter discusses the process of joining the several datasets and the pipeline we use to build the models. In total, eight datasets are used from the Vancouver Open Portal. Some of these are updated weekly by the Vancouver authorities. Therefore, the versions we used were saved and uploaded on our github, to keep the results consistent.

\section{Pre-processing}

The first two datasets used are \textit{local-area-boundary}, which contains the geometry of the several regions of Vancouver, and the second is \textit{buildings-footprints-2009}, which contains several information about buildings. The first step is to filter the buildings dataset. All the columns were removed except the building id, roof type, the highest elevation and the geometry. Several buildings shared the same id, hence these were joined. When joining, the result has the maximum highest elevation, and the union of the two buildings as geometry. The count of the join is put into a new column called sub-buildings. Lastly, since some buildings outside of the Vancouver area were included in the dataset, a check is made such that if the geometry of the building does not fully intersect with a local area, then it is discarded.

Although this dataset contains useful information about the buildings, it does not contain some important information, namely the street and local area that the building lies in. To get these two fields, we join with \textit{property-addresses}. A property does not necessarily indicate a building, but merely a plot of land. Hence, to merge this dataset with our previous one, we join the coordinate of the properties with the geometry shape of the buildings. The properties are looped on each building and we merge with the first one whose coordinate intersects the geometry of the building.

The next step is to join with the \textit{graffiti} dataset in order to get the number of graffiti per building. This was done by joining on the coordinate of the graffiti and the coordinate of the property that the building was previously joined with. If we join with the coordinates rounded to five decimal places\footnote{five decimal places is precision needed to distinguish objects up to 1.1m apart from each other, for example, 2 trees}, every graffiti will lie on a property coordinate. However, since some properties were previously pruned due to joining, we unfortunately end up losing 27.6\% of the graffiti instances.

To extract more information from our data, we also included objects which are close to each building. In terms of distances, we estimated the approximate distance of one house, two houses and four houses, and included the features of nearby buildings into the features of the current one. The new features are called : \textit{one\_house\_away\_buildings\_count,one\_house\_away\_graffiti\_count, \ldots}. Then we also added the other distances as well, so we had \textit{two\_houses\_away\_buildings\_count, \ldots, four\_houses\_away\_street\_lights}. The full feature set can be seen in the Appendix~\ref{app:all_model_features}. The street lights data came from the \textit{street-lighting-poles} dataset. Since coordinates are used,geodesic distance would have been the most appropriate distance measure to use. However, since we are working with a very small area of the planet, results end up being almost identical as using euclidean. As a result, euclidean was used due to its speed, as projection turns out to be a very resource intensive process, especially if it needs to be done quadratically to the dataset.

Leaving the resultant previous dataset on hold, we do some processing on other smaller datasets which will be joined later. First, the \textit{local-area-boundary} and \textit{cencus2016} datasets are joined to obtain the area in meters squared, the population and the population density of each local area are extracted and put into a single dataset. The \textit{public-streets} dataset is also cleaned so that it will give us the type of each street, for example, whether the street is arterial or residential.

Finally, everything is joined together. Any Vancouver specific feature is removed such as local area and street names. The street and roof type features are one-hot encoded so we can build a model out of them. Next, we discuss the pipeline used to build our regression and classification models.

\section{Models}

For our purposes, we train both regression and classification models. The target value for regression is the graffiti count for each building, while for classification we predict whether a building has or does not have graffiti on it. For each model that we build, the pipeline is the same. We first perform scaling and then perform hyper parameter tuning on a randomly selected 5000 samples of our dataset. With regards to scaling, we scaled the continuous variables to mean 0 and standard deviation 1. Next, we split the full dataset into five folds and perform cross validation. A scaler is first fit on the train set, then we train the model, then the scaler is applied to the testing set, and lastly we pass the testing set through our model. For the full model parameters which we tried, refer to Appendix~\ref{app:model_parameters}.

The results shown in the web page are the averages of the results obtained by our models. For the full results tables, refer to Appendix~\ref{app:full_results}. When it comes to explaining our models, however, we train separate linear models on our entire dataset. We then put the entire dataset back into the model to get the predictions based on the fitted parameters. Linear models were chosen for this task to minimize the risk of overfitting. These predictions were then used to generate heatmaps of the anticipated locations where new graffiti will appear.

\section{Presenting Results}
% how we will be showing stuff to user
% results + discussion (when we point out a result we also discuss it to show what it means)
  % correlation
  % FN vs FP models (which to prioritize? This might depend on the government body at hand)
  % Predict where new graffiti will be
