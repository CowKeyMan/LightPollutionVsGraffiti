\chapter{Conclusion}

% 1 paragraph summarizing what we did
In this project, we considered two tasks: a regression problem and a classification problem. Many models were trained, and while the neural network based models gave the highest accuracy, the three linear models were not far behind. As such, these were chosen to help explain which features are the most important in predicting the prevention or encouragement of graffiti. In our results, we have provided possible hypotheses as to why the identified features are so helpful for the models to be able to predict graffiti. These hypotheses can help our target audiences to come up with more useful strategies towards preventing graffiti, and to encourage further research in the area of preventing graffiti.

% 1 paragraph describing limitations and future work
With regards to future work, we would like to apply our models to other cities besides Vancouver. Furthermore, some datasets were not always up to date. For example, the buildings footprint dataset is from 2009. Our methodology does not take this into account, hence, rerunning the project with updated data may lead to results which are more representative of the current situation.
