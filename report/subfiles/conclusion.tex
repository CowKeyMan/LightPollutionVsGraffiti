\chapter{Conclusion}

% 1 paragraph summarizing what we did
We consider two tasks to solve which are the regression problem and the logistic regression. Linear regression helps us to predict the actual number of graffiti that will be on each building, meanwhile logistic regression helps us to predict whether a building will be vandalized with graffiti or not. We use the linear regression to highlight and explain the most contributing features, because it is much more interpretable than a Neural Network. Since Neural network is not drastically more accurate than the base logistic regression model, we use the logistic regression models to explain our data and provide suggestions to mitigate graffiti.

% 1 paragraph saying what we learnt
In conclusion, we learnt that, we should have more dense areas, and less buildings with very few people next to them when planning cities. Our hypothesis for this is that if there are more people in an area, there are consequently more eyes protecting it from vandalism. We also learnt that graffiti spreads similar to a virus, that is one building to the next building. In spite of this, if the nearby buildings is high, then there is less chance of the current buildings having graffiti. We form the hypothesis that this is because buildings with high graffiti density act as a ‘magnet’ for new graffiti, protecting other buildings from sharing the same fate. Furthermore, since buildings with a flat roof type attract more graffiti, we should invest in higher quality and more elaborate buildings, rather than plain ones. In the long run, this will be more cost effective since there will not be any need to clean up. We also notice that graffiti in Downtown, Mount Pleasant, and Westend is set to increase, while the southern part of Marpole is trending towards becoming  the newest hub for graffiti. As a result, it would be a good idea to target these areas when it comes to enforcement.

% 1 paragraph describing limitations and future work
Some datasets were not always up to date, for example, the buildings footprint dataset is from 2009. Our methodology does not take this into account, hence, rerunning the project with updated data may lead to more results more representative of the current situation.

% Given the same features, we would be able to do the same for different cities around the world.
Our result will be displayed in an online blog which will be aimed at officials working at the city council of Vancouver. The result will be suit perfectly for the Vancouver officials because we made a model based on the Vancouver dataset. We will also be able to do the same for different cities around the worlds if there’s similar features. However, other officials from all over the globe may also view it and adapt to their problem as well. Therefore, it may prompt a new way of dealing with graffiti before it occurs. Moreover, city planners may use this to design cities which are graffiti proof by design. Finally, to inspire other researchers to explore this area further, they can see our methodology and technical details in this report.
