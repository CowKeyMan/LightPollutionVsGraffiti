%% This file is modified by Jussi Kangasharju and Pirjo Moen.
%% Earlier versions were made by Veli Mäkinen
%% from HY_fysiikka_LuKtemplate.tex authored by Roope Halonen ja
%% Tomi Vainio. Some text is also inherited from engl_malli.tex by
%% Kutvonen, Erkiö, Mäkelä, Verkamo, Kurhila, and Nykänen.
%%
%%
% STEP 1: Choose oneside or twoside
\documentclass[english,twoside,openright]{UH_DS_report}
%finnish,swedish
%
%\usepackage[utf8]{inputenc}
% For UTF8 support. Use UTF8 when saving your file.
\usepackage{lmodern} % Font package
\usepackage{textcomp} % Package for special symbols
\usepackage[pdftex]{color, graphicx} % For pdf output and jpg/png graphics
\usepackage[pdftex, plainpages=false]{hyperref} % For hyperlinks and pdf metadata
\usepackage{fancyhdr} % For nicer page headers
\usepackage{tikz} % For making vector graphics (hard to learn but powerful)
%\usepackage{wrapfig} % For nice text-wrapping figures (use at own discretion)
\usepackage{amsmath, amssymb} % For better math
%\usepackage[square]{natbib} % For bibliography
\usepackage[footnotesize,bf]{caption} % For more control over figure captions
\usepackage{blindtext}
\usepackage{titlesec}
\usepackage[titletoc]{appendix}

\onehalfspacing %line spacing
%\singlespacing
%\doublespacing

%\fussy
\sloppy % sloppy and fussy commands can be used to avoid overlong text lines

% STEP 2:
% Set up all the information for the title page and the abstract form.
% Replace parameters with your information.
\title{Predicting Graffiti using City Layouts}
\author{Adiel Lindroos, Ari Barkhah, Daniel Cauchi}
\date{\today}
%\prof{Professor X or Dr. Y}
%\censors{Professor A}{Dr. B}{}
\keywords{}
\depositeplace{}
\additionalinformation{}


\classification{\protect{\ \\
		 \  General and reference $\rightarrow$ Document types $\rightarrow$ Surveys and overviews\  \\
		\  Applied computing $\rightarrow$ Document management and text processing $\rightarrow$ Document management $\rightarrow$ Text editing\\
}}

% if you want to quote someone special. You can comment this line and
% there will be nothing on the document.
%\quoting{Bachelor's degrees make pretty good placemats if you get them
%laminated.}{Jeph Jacques}

% OPTIONAL STEP: Set up properties and metadata for the pdf file that
% pdfLaTeX makes. But you don't really need to do this unless you want
% to.
\hypersetup{
	%bookmarks=true,         % show bookmarks bar first?
	unicode=true,           % to show non-Latin characters in Acrobat’s bookmarks
	pdftoolbar=true,        % show Acrobat’s toolbar?
	pdfmenubar=true,        % show Acrobat’s menu?
	pdffitwindow=false,		% window fit to page when opened
	pdfstartview={FitH},    % fits the width of the page to the window
	pdftitle={},            % title
	pdfauthor={},           % author
	pdfsubject={},          % subject of the document
	pdfcreator={},          % creator of the document
	pdfproducer={pdfLaTeX}, % producer of the document
	pdfkeywords={something} {something else}, % list of keywords for
	pdfnewwindow=true,      % links in new window
	colorlinks=true, 		% false: boxed links; true: colored links
	linkcolor=black,        % color of internal links
	citecolor=black,        % color of links to bibliography
	filecolor=magenta,      % color of file links
	urlcolor=cyan			% color of external links
}

\begin{document}

% Generate title page.
\maketitle

% STEP 3: Write your abstract (of course you really do this last). You
% can make several abstract pages (if you want it in different
% languages), but you should also then redefine some of the above
% parameters in the proper language as well, in between the abstract
% definitions.
\begin{abstract}

Graffiti is a global problem which is costing governments billions of dollars each year for cleanup and repair costs. It can also lead to other negative effects such as discouraging tourism and damaging the surface it is defacing. In this project we conduct what is, to the best of our knowledge, the first ever study about methods to predict and prevent graffiti using the features of a city. Our results show that our methods are promising. While future work may uncover better features in predicting this act of vandalism, our methods are a good starting point and can already be used to show hypothetical relationships between the chosen city features and the generation of graffiti.

\end{abstract}


% Place ToC
\mytableofcontents

\mynomenclature

% ----------------------------------------------------------------------
% STEP 4: Write the report. Your actual text starts here.
% You shouldn't mess with the code above the line except to change the
% parameters. Removing the abstract and ToC commands will mess up stuff.
\chapter{Introduction}

Define problem

What we started with and initial problem

What we are ended up doing (briefly)

Say outline of report
 % 1 pg

\chapter{Methodology}

% mention how graffiti with light would not work, say why we opted for buildings

This chapter discusses the process of joining the several datasets and the pipeline we use to build the models. In total, eight datasets are used from the Vancouver Open Portal. Some of these are updated weekly by the Vancouver authorities. Therefore, the versions we used were saved and uploaded on our github, to keep the results consistent.

\section{Pre-processing}

The first two datasets used are \textit{local-area-boundary}, which contains the geometry of the several regions of Vancouver, and the second is \textit{buildings-footprints-2009}, which contains several information about buildings. The first step is to filter the buildings dataset. All the columns were removed except the building id, roof type, the highest elevation and the geometry. Several buildings shared the same id, hence these were joined. When joining, the result has the maximum highest elevation, and the union of the two buildings as geometry. The count of the join is put into a new column called sub-buildings. Lastly, since some buildings outside of the Vancouver area were included in the dataset, a check is made such that if the geometry of the building does not fully intersect with a local area, then it is discarded.

Although this dataset contains useful information about the buildings, it does not contain some important information, namely the street and local area that the building lies in. To get these two fields, we join with \textit{property-addresses}. A property does not necessarily indicate a building, but merely a plot of land. Hence, to merge this dataset with our previous one, we join the coordinate of the properties with the geometry shape of the buildings. The properties are looped on each building and we merge with the first one whose coordinate intersects the geometry of the building.

The next step is to join with the \textit{graffiti} dataset in order to get the number of graffiti per building. This was done by joining on the coordinate of the graffiti and the coordinate of the property that the building was previously joined with. If we join with the coordinates rounded to five decimal places\footnote{five decimal places is precision needed to distinguish objects up to 1.1m apart from each other, for example, 2 trees}, every graffiti will lie on a property coordinate. However, since some properties were previously pruned due to joining, we unfortunately end up losing 27.6\% of the graffiti instances.

To extract more information from our data, we also included objects which are close to each building. In terms of distances, we estimated the approximate distance of one house, two houses and four houses, and included the features of nearby buildings into the features of the current one. The new features are called : \textit{one\_house\_away\_buildings\_count,one\_house\_away\_graffiti\_count, \ldots}. Then we also added the other distances as well, so we had \textit{two\_houses\_away\_buildings\_count, \ldots, four\_houses\_away\_street\_lights}. The full feature set can be seen in the Appendix~\ref{app:all_model_features}. The street lights data came from the \textit{street-lighting-poles} dataset. Since coordinates are used,geodesic distance would have been the most appropriate distance measure to use. However, since we are working with a very small area of the planet, results end up being almost identical as using euclidean. As a result, euclidean was used due to its speed, as projection turns out to be a very resource intensive process, especially if it needs to be done quadratically to the dataset.

Leaving the resultant previous dataset on hold, we do some processing on other smaller datasets which will be joined later. First, the \textit{local-area-boundary} and \textit{cencus2016} datasets are joined to obtain the area in meters squared, the population and the population density of each local area are extracted and put into a single dataset. The \textit{public-streets} dataset is also cleaned so that it will give us the type of each street, for example, whether the street is arterial or residential.

Finally, everything is joined together. Any Vancouver specific feature is removed such as local area and street names. The street and roof type features are one-hot encoded so we can build a model out of them. Next, we discuss the pipeline used to build our regression and classification models.

\section{Models}

For our purposes, we train both regression and classification models. The target value for regression is the graffiti count for each building, while for classification we predict whether a building has or does not have graffiti on it. For each model that we build, the pipeline is the same. We first perform scaling and then perform hyper parameter tuning on a randomly selected 5000 samples of our dataset. With regards to scaling, we scaled the continuous variables to mean 0 and standard deviation 1. Next, we split the full dataset into five folds and perform cross validation. A scaler is first fit on the train set, then we train the model, then the scaler is applied to the testing set, and lastly we pass the testing set through our model. For the full model parameters which we tried, refer to Appendix~\ref{app:model_parameters}.

The results shown in the web page are the averages of the results obtained by our models. For the full results tables, refer to Appendix~\ref{app:full_results}. When it comes to explaining our models, however, we train separate linear models on our entire dataset. We then put the entire dataset back into the model to get the predictions based on the fitted parameters. Linear models were chosen for this task to minimize the risk of overfitting. These predictions were then used to generate heatmaps of the anticipated locations where new graffiti will appear.
 % 1.5 pgs
\chapter{Outcomes}

\section{Project Outcomes}
% results + discussion (when we point out a result we also discuss it to show what it means)
  % correlation
  % FN vs FP models (which to prioritize? This might depend on the government body at hand)
  % Predict where new graffiti will be


\section{Learning Outcomes} % .5 pages, maybe less

%
  % 2 pgs

\chapter{Conclusion}

% 1 paragraph summarizing what we did


% 1 paragraph describing limitations and future work
Some datasets were not always up to date, for example, the buildings footprint dataset is from 2009. Our methodology does not take this into account, hence, rerunning the project with updated data may lead to more results more representative of the current situation.
 % .5 pgs

\cleardoublepage %fixes the position of bibliography in bookmarks
\phantomsection

\renewcommand\bibname{References}
\addcontentsline{toc}{chapter}{\bibname} % This lines adds the bibliography to the ToC
\bibliographystyle{abbrv} % numbering alphabetic order
\bibliography{main}

\begin{appendices}
\myappendixtitle
\chapter{All Model Features}\label{app:all_model_features}

\begin{itemize}
  \item highest\_elevation\_m
  \item area\_m2
  \item sub\_buildings
  \item one\_house\_away\_buildings\_count
  \item one\_house\_away\_graffiti\_count
  \item one\_house\_away\_graffiti\_average
  \item one\_house\_away\_graffiti\_buildings
  \item one\_house\_away\_buildings\_average\_height
  \item one\_house\_away\_buildings\_median\_height
  \item one\_house\_away\_buildings\_total\_sub\_buildings
  \item one\_house\_away\_buildings\_average\_sub\_buildings
  \item one\_house\_away\_buildings\_median\_sub\_buildings
  \item one\_house\_away\_street\_lights
  \item two\_houses\_away\_buildings\_count
  \item two\_houses\_away\_graffiti\_count
  \item two\_houses\_away\_graffiti\_average
  \item two\_houses\_away\_graffiti\_buildings
  \item two\_houses\_away\_buildings\_average\_height
  \item two\_houses\_away\_buildings\_median\_height
  \item two\_houses\_away\_buildings\_total\_sub\_buildings
  \item two\_houses\_away\_buildings\_average\_sub\_buildings
  \item two\_houses\_away\_buildings\_median\_sub\_buildings
  \item two\_houses\_away\_street\_lights
  \item four\_houses\_away\_buildings\_count
  \item four\_houses\_away\_graffiti\_count
  \item four\_houses\_away\_graffiti\_average
  \item four\_houses\_away\_graffiti\_buildings
  \item four\_houses\_away\_buildings\_average\_height
  \item four\_houses\_away\_buildings\_median\_height
  \item four\_houses\_away\_buildings\_total\_sub\_buildings
  \item four\_houses\_away\_buildings\_average\_sub\_buildings
  \item four\_houses\_away\_buildings\_median\_sub\_buildings
  \item four\_houses\_away\_street\_lights
  \item geo\_local\_area\_area\_m2
  \item geo\_local\_area\_population
  \item pop\_density
  \item roof\_type\_Complex
  \item roof\_type\_Flat
  \item roof\_type\_Pitched
  \item street\_type\_arterial
  \item street\_type\_collector
  \item street\_type\_residential
  \item street\_type\_secondary\_arterial
\end{itemize}

\chapter{Model Parameters Code}\label{app:model_parameters}

This appendix shows the model parameters options used as python code. The format was purposefully chosen to be similar to the JSON format in case this was to be ported to a configuration file.

\begin{Verbatim}

pow_10_paramter = [0.0001, 0.001, 0.01, 0.1, 1, 10, 100]

regression_models = {
    "LinearRegression": {
        "class": sklearn.linear_model.LinearRegression,
        "hyperparameters": {},
        "set_parameters": {
            "n_jobs": -1,
        }
    },
    "Ridge": {
        "class": sklearn.linear_model.Ridge,
        "hyperparameters": {
            "alpha": pow_10_paramter,
            "tol": pow_10_paramter,
            "solver": [
                "auto",
                "svd",
                "cholesky",
                "lsqr",
                "sparse_cg",
                "sag",
                "saga",
            ]
        },
        "set_parameters": {}
    },
    "Lasso": {
        "class": sklearn.linear_model.Lasso,
        "hyperparameters": {
            "alpha": pow_10_paramter,
        },
        "set_parameters": {
            "tol": 0.0001,
            "max_iter": 10000
        }
    },
    "MLPRegressor": {
        "class": sklearn.neural_network.MLPRegressor,
        "hyperparameters": {
            "hidden_layer_sizes": [
                (200, 200), (100, 100), (100, 100, 100), (200, 100, 100)
            ],
            "activation": ["identity", "logistic", "tanh", "relu"],
            "solver": ["lbfgs", "sgd", "adam"],
            "alpha": pow_10_paramter,
        },
        "set_parameters": {
            "early_stopping": True,
            "max_iter": 500,
        }
    }
}

classification_models = {
    "LogisticRegression": {
        "class": sklearn.linear_model.LogisticRegression,
        "hyperparameters": {
            "C": pow_10_paramter,
        },
        "set_parameters": {
            "n_jobs": -1,
            "max_iter": 200
        }
    },
    "BalancedLogisticRegression": {
        "class": sklearn.linear_model.LogisticRegression,
        "hyperparameters": {
            "C": pow_10_paramter,
        },
        "set_parameters": {
            "n_jobs": -1,
            "max_iter": 200,
            "class_weight": "balanced"
        }
    },
    "MLPClassifier": {
        "class": sklearn.neural_network.MLPClassifier,
        "hyperparameters": {
            "hidden_layer_sizes": [
                (200, 200), (100, 100), (100, 100, 100), (200, 100, 100)
            ],
            "activation": ["identity", "logistic", "tanh", "relu"],
            "solver": ["lbfgs", "sgd", "adam"],
            "alpha": pow_10_paramter,
        },
        "set_parameters": {
            "early_stopping": True,
            "max_iter": 500,
        }
    }
}

\end{Verbatim}

\chapter{Full Results}\label{app:full_results}

This appendix shows the full results of the trained models. First we show the results of the regression models and then we show those of the classification models.

\section{Regression Models}
\begin{Verbatim}
LinearRegression
	Best Params: {'n_jobs': -1}
	Fitting fold 1/5
		r^2 on train: 0.604810872501663
		r^2 on test: 0.602735617491803
		r^2 on test not 0: 0.40932809270610593

	Fitting fold 2/5
		r^2 on train: 0.6057825097767984
		r^2 on test: 0.5983170039845684
		r^2 on test not 0: 0.37047783129055967

	Fitting fold 3/5
		r^2 on train: 0.5938622236974815
		r^2 on test: 0.637858310082432
		r^2 on test not 0: 0.42882042181419155

	Fitting fold 4/5
		r^2 on train: 0.6143160363114665
		r^2 on test: 0.556919897520383
		r^2 on test not 0: 0.3427041111951884

	Fitting fold 5/5
		r^2 on train: 0.606295950014363
		r^2 on test: 0.5962737548077484
		r^2 on test not 0: 0.3466456468442167

Ridge
	Best Params: {
		'tol': 10, 'solver': 'saga', 'alpha': 0.1
	}
	Fitting fold 1/5
		r^2 on train: 0.5959033693753599
		r^2 on test: 0.5897855828772042
		r^2 on test not 0: 0.29353447003963984

	Fitting fold 2/5
		r^2 on train: 0.5944071266025769
		r^2 on test: 0.6132781891163086
		r^2 on test not 0: 0.397107616253884

	Fitting fold 3/5
		r^2 on train: 0.5963030563797831
		r^2 on test: 0.6178928473501317
		r^2 on test not 0: 0.43354017996331506

	Fitting fold 4/5
		r^2 on train: 0.6021928081960923
		r^2 on test: 0.5681189460151417
		r^2 on test not 0: 0.3093194836548464

	Fitting fold 5/5
		r^2 on train: 0.6032687634903349
		r^2 on test: 0.5873165844909658
		r^2 on test not 0: 0.39821716410681984

Lasso
	Best Params: {
		'alpha': 0.0001, 'tol': 0.0001, 'max_iter': 10000
	}
	Fitting fold 1/5
		r^2 on train: 0.601668565902004
		r^2 on test: 0.615154345096635
		r^2 on test not 0: 0.4012183382401111

	Fitting fold 2/5
		r^2 on train: 0.6105152645125385
		r^2 on test: 0.5769134698563446
		r^2 on test not 0: 0.33234511018722546

	Fitting fold 3/5
		r^2 on train: 0.6030295669805608
		r^2 on test: 0.6088369388640831
		r^2 on test not 0: 0.38269822309692536

	Fitting fold 4/5
		r^2 on train: 0.6016016089606414
		r^2 on test: 0.6119293483115629
		r^2 on test not 0: 0.4264429862940726

	Fitting fold 5/5
		r^2 on train: 0.608313883247569
		r^2 on test: 0.5881777101384241
		r^2 on test not 0: 0.3663869068730805

MLPRegressor
	Best Params: {
		'solver': 'sgd',
		'hidden_layer_sizes': (100, 100),
		'alpha': 1,
		'activation': 'relu',
		'early_stopping': True,
		'max_iter': 500
	  }
	Fitting fold 1/5
		r^2 on train: 0.6583630475407372
		r^2 on test: 0.6023958144772487
		r^2 on test not 0: 0.4238171245000216

	Fitting fold 2/5
		r^2 on train: 0.6299781799908495
		r^2 on test: 0.6413926930989139
		r^2 on test not 0: 0.4186273581481599

	Fitting fold 3/5
		r^2 on train: 0.6340920042110348
		r^2 on test: 0.637685957735554
		r^2 on test not 0: 0.4153368443725123

	Fitting fold 4/5
		r^2 on train: 0.6375479497238612
		r^2 on test: 0.6308132115181082
		r^2 on test not 0: 0.4353551223538782

	Fitting fold 5/5
		r^2 on train: 0.6539979725087999
		r^2 on test: 0.6097699517695414
		r^2 on test not 0: 0.37628497918505943
\end{Verbatim}

\section{Regression Models}
\begin{Verbatim}
LogisticRegression
	Best Params: {'C': 0.1, 'n_jobs': -1, 'max_iter': 200}
	Fitting fold 1/5
		accuracy on train: 0.9596210511889262
		f1 on train: 0.7107620566940728
		accuracy on test: 0.9586788186116632
		precision on test: 0.8077777777777778
		recall on test: 0.6283491789109766
		f1 on test: 0.7068546426835197

	Fitting fold 2/5
		accuracy on train: 0.9595353936819023
		f1 on train: 0.7096853490658801
		accuracy on test: 0.9595696566847118
		precision on test: 0.8210645526613817
		recall on test: 0.6266205704407951
		f1 on test: 0.7107843137254902

	Fitting fold 3/5
		accuracy on train: 0.9595011306790927
		f1 on train: 0.710009813542689
		accuracy on test: 0.9590899746453779
		precision on test: 0.819634703196347
		recall on test: 0.6205704407951599
		f1 on test: 0.7063453025086079

	Fitting fold 4/5
		accuracy on train: 0.959792366202974
		f1 on train: 0.7122716685055781
		accuracy on test: 0.9577879805386144
		precision on test: 0.8141695702671312
		recall on test: 0.6058772687986171
		f1 on test: 0.6947472745292369

	Fitting fold 5/5
		accuracy on train: 0.9589700541355445
		f1 on train: 0.7061710219604956
		accuracy on test: 0.9619680668813815
		precision on test: 0.8380202474690663
		recall on test: 0.6444636678200693
		f1 on test: 0.7286063569682152

BalancedLogisticRegression
	Best Params: {
		'C': 0.0001,
		'n_jobs': -1,
		'max_iter': 200,
		'class_weight': 'balanced'
	}
	Fitting fold 1/5
		accuracy on train: 0.9327245939834167
		f1 on train: 0.6649603276170974
		accuracy on test: 0.9335983005550607
		precision on test: 0.5529279279279279
		recall on test: 0.8487467588591184
		f1 on test: 0.6696215479031709

	Fitting fold 2/5
		accuracy on train: 0.9317823614061537
		f1 on train: 0.6615096905814349
		accuracy on test: 0.9329815665044885
		precision on test: 0.5492028587135789
		recall on test: 0.8634399308556612
		f1 on test: 0.6713709677419354

	Fitting fold 3/5
		accuracy on train: 0.9335297745494415
		f1 on train: 0.6682626538987688
		accuracy on test: 0.9316110463921058
		precision on test: 0.544839255499154
		recall on test: 0.8349178910976663
		f1 on test: 0.6593856655290102

	Fitting fold 4/5
		accuracy on train: 0.9337867470705132
		f1 on train: 0.6691774373020628
		accuracy on test: 0.9296237922291509
		precision on test: 0.5359911406423035
		recall on test: 0.8366464995678479
		f1 on test: 0.6533918326020925

	Fitting fold 5/5
		accuracy on train: 0.9327588569862263
		f1 on train: 0.6661563323977205
		accuracy on test: 0.9364763927910642
		precision on test: 0.567551622418879
		recall on test: 0.8321799307958477
		f1 on test: 0.6748509294984215

MLPClassifier
	Best Params: {
		'solver': 'sgd',
		'hidden_layer_sizes': (200, 200),
		'alpha': 1, 'activation': 'tanh',
		'early_stopping': True,
		'max_iter': 500
	}
	Fitting fold 1/5
		accuracy on train: 0.9587987391214966
		f1 on train: 0.7154856264048267
		accuracy on test: 0.9610772288083328
		precision on test: 0.8058151609553479
		recall on test: 0.6707000864304236
		f1 on test: 0.7320754716981133

	Fitting fold 2/5
		accuracy on train: 0.9607517302816418
		f1 on train: 0.7251349730053989
		accuracy on test: 0.9572397724936613
		precision on test: 0.7971014492753623
		recall on test: 0.6179775280898876
		f1 on test: 0.6962025316455697

	Fitting fold 3/5
		accuracy on train: 0.9598094977043788
		f1 on train: 0.7161181026137463
		accuracy on test: 0.9610087028027137
		precision on test: 0.8188720173535792
		recall on test: 0.6525496974935178
		f1 on test: 0.7263107263107264

	Fitting fold 4/5
		accuracy on train: 0.960186390735284
		f1 on train: 0.7205387205387205
		accuracy on test: 0.9598437607071884
		precision on test: 0.8073196986006459
		recall on test: 0.648228176318064
		f1 on test: 0.7190795781399808

	Fitting fold 5/5
		accuracy on train: 0.960032207222641
		f1 on train: 0.7212331222368265
		accuracy on test: 0.9586788186116632
		precision on test: 0.791974656810982
		recall on test: 0.6487889273356401
		f1 on test: 0.7132667617689015
\end{Verbatim}

\end{appendices}

\end{document}
