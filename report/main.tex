%% This file is modified by Jussi Kangasharju and Pirjo Moen.
%% Earlier versions were made by Veli Mäkinen
%% from HY_fysiikka_LuKtemplate.tex authored by Roope Halonen ja
%% Tomi Vainio. Some text is also inherited from engl_malli.tex by
%% Kutvonen, Erkiö, Mäkelä, Verkamo, Kurhila, and Nykänen.
%%
%%
% STEP 1: Choose oneside or twoside
\documentclass[english,twoside,openright]{UH_DS_report}
%finnish,swedish
%
%\usepackage[utf8]{inputenc}
% For UTF8 support. Use UTF8 when saving your file.
\usepackage{lmodern} % Font package
\usepackage{textcomp} % Package for special symbols
\usepackage[pdftex]{color, graphicx} % For pdf output and jpg/png graphics
\usepackage[pdftex, plainpages=false]{hyperref} % For hyperlinks and pdf metadata
\usepackage{fancyhdr} % For nicer page headers
\usepackage{tikz} % For making vector graphics (hard to learn but powerful)
%\usepackage{wrapfig} % For nice text-wrapping figures (use at own discretion)
\usepackage{amsmath, amssymb} % For better math
%\usepackage[square]{natbib} % For bibliography
\usepackage[footnotesize,bf]{caption} % For more control over figure captions
\usepackage{blindtext}
\usepackage{titlesec}
\usepackage[titletoc]{appendix}

\onehalfspacing %line spacing
%\singlespacing
%\doublespacing

%\fussy
\sloppy % sloppy and fussy commands can be used to avoid overlong text lines

% STEP 2:
% Set up all the information for the title page and the abstract form.
% Replace parameters with your information.
\title{Predicting Graffiti using City Layouts}
\author{Adiel Lindroos, Ari Barkhah, Daniel Cauchi}
\date{\today}
%\prof{Professor X or Dr. Y}
%\censors{Professor A}{Dr. B}{}
\keywords{}
\depositeplace{}
\additionalinformation{}


\classification{\protect{\ \\
		 \  General and reference $\rightarrow$ Document types $\rightarrow$ Surveys and overviews\  \\
		\  Applied computing $\rightarrow$ Document management and text processing $\rightarrow$ Document management $\rightarrow$ Text editing\\
}}

% if you want to quote someone special. You can comment this line and
% there will be nothing on the document.
%\quoting{Bachelor's degrees make pretty good placemats if you get them
%laminated.}{Jeph Jacques}

% OPTIONAL STEP: Set up properties and metadata for the pdf file that
% pdfLaTeX makes. But you don't really need to do this unless you want
% to.
\hypersetup{
	%bookmarks=true,         % show bookmarks bar first?
	unicode=true,           % to show non-Latin characters in Acrobat’s bookmarks
	pdftoolbar=true,        % show Acrobat’s toolbar?
	pdfmenubar=true,        % show Acrobat’s menu?
	pdffitwindow=false,		% window fit to page when opened
	pdfstartview={FitH},    % fits the width of the page to the window
	pdftitle={},            % title
	pdfauthor={},           % author
	pdfsubject={},          % subject of the document
	pdfcreator={},          % creator of the document
	pdfproducer={pdfLaTeX}, % producer of the document
	pdfkeywords={something} {something else}, % list of keywords for
	pdfnewwindow=true,      % links in new window
	colorlinks=true, 		% false: boxed links; true: colored links
	linkcolor=black,        % color of internal links
	citecolor=black,        % color of links to bibliography
	filecolor=magenta,      % color of file links
	urlcolor=cyan			% color of external links
}

\begin{document}

% Generate title page.
\maketitle

% STEP 3: Write your abstract (of course you really do this last). You
% can make several abstract pages (if you want it in different
% languages), but you should also then redefine some of the above
% parameters in the proper language as well, in between the abstract
% definitions.
\begin{abstract}

Graffiti is a global problem which is costing governments billions of dollars each year for cleanup and repair costs. It can also lead to other negative effects such as discouraging tourism and damaging the surface it is defacing. In this project we conduct what is, to the best of our knowledge, the first ever study about methods to predict and prevent graffiti using the features of a city. Our results show that our methods are promising. While future work may uncover better features in predicting this act of vandalism, our methods are a good starting point and can already be used to show hypothetical relationships between the chosen city features and the generation of graffiti.

\end{abstract}


% Place ToC
\mytableofcontents

\mynomenclature

% ----------------------------------------------------------------------
% STEP 4: Write the report. Your actual text starts here.
% You shouldn't mess with the code above the line except to change the
% parameters. Removing the abstract and ToC commands will mess up stuff.
\chapter{Introduction}

Define problem

What we started with and initial problem

What we are ended up doing (briefly)

Say outline of report



\chapter{Conclusion}

% 1 paragraph summarizing what we did


% 1 paragraph describing limitations and future work
Some datasets were not always up to date, for example, the buildings footprint dataset is from 2009. Our methodology does not take this into account, hence, rerunning the project with updated data may lead to more results more representative of the current situation.


\cleardoublepage %fixes the position of bibliography in bookmarks
\phantomsection

\renewcommand\bibname{References}
\addcontentsline{toc}{chapter}{\bibname} % This lines adds the bibliography to the ToC
\bibliographystyle{abbrv} % numbering alphabetic order
\bibliography{main}

\begin{appendices}
\myappendixtitle
\end{appendices}

\end{document}
