%% This file is modified by Jussi Kangasharju and Pirjo Moen.
%% Earlier versions were made by Veli Mäkinen
%% from HY_fysiikka_LuKtemplate.tex authored by Roope Halonen ja
%% Tomi Vainio. Some text is also inherited from engl_malli.tex by
%% Kutvonen, Erkiö, Mäkelä, Verkamo, Kurhila, and Nykänen.
%%
%%
% STEP 1: Choose oneside or twoside
\documentclass[english,twoside,openright]{UH_DS_report}
%finnish,swedish
%
%\usepackage[utf8]{inputenc}
% For UTF8 support. Use UTF8 when saving your file.
\usepackage{lmodern} % Font package
\usepackage{textcomp} % Package for special symbols
\usepackage[pdftex]{color, graphicx} % For pdf output and jpg/png graphics
\usepackage[pdftex, plainpages=false]{hyperref} % For hyperlinks and pdf metadata
\usepackage{fancyhdr} % For nicer page headers
\usepackage{tikz} % For making vector graphics (hard to learn but powerful)
%\usepackage{wrapfig} % For nice text-wrapping figures (use at own discretion)
\usepackage{amsmath, amssymb} % For better math
%\usepackage[square]{natbib} % For bibliography
\usepackage[footnotesize,bf]{caption} % For more control over figure captions
\usepackage{blindtext}
\usepackage{titlesec}
\usepackage[titletoc]{appendix}

\onehalfspacing %line spacing
%\singlespacing
%\doublespacing

%\fussy
\sloppy % sloppy and fussy commands can be used to avoid overlong text lines

% STEP 2:
% Set up all the information for the title page and the abstract form.
% Replace parameters with your information.
\title{Predicting Graffiti using City Layouts}
\author{Adiel Lindroos, Ari Barkhah, Daniel Cauchi}
\date{\today}
%\prof{Professor X or Dr. Y}
%\censors{Professor A}{Dr. B}{}
\keywords{}
\depositeplace{}
\additionalinformation{}


\classification{\protect{\ \\
		 \  General and reference $\rightarrow$ Document types $\rightarrow$ Surveys and overviews\  \\
		\  Applied computing $\rightarrow$ Document management and text processing $\rightarrow$ Document management $\rightarrow$ Text editing\\
}}

% if you want to quote someone special. You can comment this line and
% there will be nothing on the document.
%\quoting{Bachelor's degrees make pretty good placemats if you get them
%laminated.}{Jeph Jacques}

% OPTIONAL STEP: Set up properties and metadata for the pdf file that
% pdfLaTeX makes. But you don't really need to do this unless you want
% to.
\hypersetup{
	%bookmarks=true,         % show bookmarks bar first?
	unicode=true,           % to show non-Latin characters in Acrobat’s bookmarks
	pdftoolbar=true,        % show Acrobat’s toolbar?
	pdfmenubar=true,        % show Acrobat’s menu?
	pdffitwindow=false,		% window fit to page when opened
	pdfstartview={FitH},    % fits the width of the page to the window
	pdftitle={},            % title
	pdfauthor={},           % author
	pdfsubject={},          % subject of the document
	pdfcreator={},          % creator of the document
	pdfproducer={pdfLaTeX}, % producer of the document
	pdfkeywords={something} {something else}, % list of keywords for
	pdfnewwindow=true,      % links in new window
	colorlinks=true, 		% false: boxed links; true: colored links
	linkcolor=black,        % color of internal links
	citecolor=black,        % color of links to bibliography
	filecolor=magenta,      % color of file links
	urlcolor=cyan			% color of external links
}

\begin{document}

% Generate title page.
\maketitle

% STEP 3: Write your abstract (of course you really do this last). You
% can make several abstract pages (if you want it in different
% languages), but you should also then redefine some of the above
% parameters in the proper language as well, in between the abstract
% definitions.
\begin{abstract}

Graffiti is a global problem which is costing governments billions of dollars each year for cleanup and repair costs. It can also lead to other negative effects such as discouraging tourism and damaging the surface it is defacing. In this project we conduct what is, to the best of our knowledge, the first ever study about methods to predict and prevent graffiti using the features of a city. Our results show that our methods are promising. While future work may uncover better features in predicting this act of vandalism, our methods are a good starting point and can already be used to show hypothetical relationships between the chosen city features and the generation of graffiti.

\end{abstract}


% Place ToC
\mytableofcontents

\mynomenclature

% ----------------------------------------------------------------------
% STEP 4: Write the report. Your actual text starts here.
% You shouldn't mess with the code above the line except to change the
% parameters. Removing the abstract and ToC commands will mess up stuff.
\chapter{Introduction}

Define problem

What we started with and initial problem

What we are ended up doing (briefly)

Say outline of report
 % 1 pg

\chapter{Methodology}

% mention how graffiti with light would not work, say why we opted for buildings

This chapter discusses the process of joining the several datasets and the pipeline we use to build the models. In total, eight datasets are used from the Vancouver Open Portal. Some of these are updated weekly by the Vancouver authorities. Therefore, the versions we used were saved and uploaded on our github, to keep the results consistent.

\section{Pre-processing}

The first two datasets used are \textit{local-area-boundary}, which contains the geometry of the several regions of Vancouver, and the second is \textit{buildings-footprints-2009}, which contains several information about buildings. The first step is to filter the buildings dataset. All the columns were removed except the building id, roof type, the highest elevation and the geometry. Several buildings shared the same id, hence these were joined. When joining, the result has the maximum highest elevation, and the union of the two buildings as geometry. The count of the join is put into a new column called sub-buildings. Lastly, since some buildings outside of the Vancouver area were included in the dataset, a check is made such that if the geometry of the building does not fully intersect with a local area, then it is discarded.

Although this dataset contains useful information about the buildings, it does not contain some important information, namely the street and local area that the building lies in. To get these two fields, we join with \textit{property-addresses}. A property does not necessarily indicate a building, but merely a plot of land. Hence, to merge this dataset with our previous one, we join the coordinate of the properties with the geometry shape of the buildings. The properties are looped on each building and we merge with the first one whose coordinate intersects the geometry of the building.

The next step is to join with the \textit{graffiti} dataset in order to get the number of graffiti per building. This was done by joining on the coordinate of the graffiti and the coordinate of the property that the building was previously joined with. If we join with the coordinates rounded to five decimal places\footnote{five decimal places is precision needed to distinguish objects up to 1.1m apart from each other, for example, 2 trees}, every graffiti will lie on a property coordinate. However, since some properties were previously pruned due to joining, we unfortunately end up losing 27.6\% of the graffiti instances.

To extract more information from our data, we also included objects which are close to each building. In terms of distances, we estimated the approximate distance of one house, two houses and four houses, and included the features of nearby buildings into the features of the current one. The new features are called : \textit{one\_house\_away\_buildings\_count,one\_house\_away\_graffiti\_count, \ldots}. Then we also added the other distances as well, so we had \textit{two\_houses\_away\_buildings\_count, \ldots, four\_houses\_away\_street\_lights}. The full feature set can be seen in the Appendix~\ref{app:all_model_features}. The street lights data came from the \textit{street-lighting-poles} dataset. Since coordinates are used,geodesic distance would have been the most appropriate distance measure to use. However, since we are working with a very small area of the planet, results end up being almost identical as using euclidean. As a result, euclidean was used due to its speed, as projection turns out to be a very resource intensive process, especially if it needs to be done quadratically to the dataset.

Leaving the resultant previous dataset on hold, we do some processing on other smaller datasets which will be joined later. First, the \textit{local-area-boundary} and \textit{cencus2016} datasets are joined to obtain the area in meters squared, the population and the population density of each local area are extracted and put into a single dataset. The \textit{public-streets} dataset is also cleaned so that it will give us the type of each street, for example, whether the street is arterial or residential.

Finally, everything is joined together. Any Vancouver specific feature is removed such as local area and street names. The street and roof type features are one-hot encoded so we can build a model out of them. Next, we discuss the pipeline used to build our regression and classification models.

\section{Models}

For our purposes, we train both regression and classification models. The target value for regression is the graffiti count for each building, while for classification we predict whether a building has or does not have graffiti on it. For each model that we build, the pipeline is the same. We first perform scaling and then perform hyper parameter tuning on a randomly selected 5000 samples of our dataset. With regards to scaling, we scaled the continuous variables to mean 0 and standard deviation 1. Next, we split the full dataset into five folds and perform cross validation. A scaler is first fit on the train set, then we train the model, then the scaler is applied to the testing set, and lastly we pass the testing set through our model. For the full model parameters which we tried, refer to Appendix~\ref{app:model_parameters}.

The results shown in the web page are the averages of the results obtained by our models. For the full results tables, refer to Appendix~\ref{app:full_results}. When it comes to explaining our models, however, we train separate linear models on our entire dataset. We then put the entire dataset back into the model to get the predictions based on the fitted parameters. Linear models were chosen for this task to minimize the risk of overfitting. These predictions were then used to generate heatmaps of the anticipated locations where new graffiti will appear.
 % 1.5 pgs
\chapter{Outcomes}

\section{Project Outcomes}
% results + discussion (when we point out a result we also discuss it to show what it means)
  % correlation
  % FN vs FP models (which to prioritize? This might depend on the government body at hand)
  % Predict where new graffiti will be


\section{Learning Outcomes} % .5 pages, maybe less

%
  % 2 pgs

\chapter{Conclusion}

% 1 paragraph summarizing what we did


% 1 paragraph describing limitations and future work
Some datasets were not always up to date, for example, the buildings footprint dataset is from 2009. Our methodology does not take this into account, hence, rerunning the project with updated data may lead to more results more representative of the current situation.
 % .5 pgs

\cleardoublepage %fixes the position of bibliography in bookmarks
\phantomsection

\renewcommand\bibname{References}
\addcontentsline{toc}{chapter}{\bibname} % This lines adds the bibliography to the ToC
\bibliographystyle{abbrv} % numbering alphabetic order
\bibliography{main}

\begin{appendices}
\myappendixtitle
\end{appendices}

\end{document}
